%%%%%%%%%%%%%%%%%%%%%%%%%%%%%%%%%%%%%%%%%%%%%%%%%%%%%%%%%%%%%%%%%%%%%%%%%%%%
%% Trim Size : 11in x 8.5in
%% Text Area : 9.6in (include Runningheads) x 7in
%% ws-jai.tex, 26 April 2012
%% Tex file to use with ws-jai.cls written in Latex2E.
%% The content, structure, format and layout of this style file is the
%% property of World Scientific Publishing Co. Pte. Ltd.
%%%%%%%%%%%%%%%%%%%%%%%%%%%%%%%%%%%%%%%%%%%%%%%%%%%%%%%%%%%%%%%%%%%%%%%%%%%%
%%

%\documentclass[draft]{ws-jai}
\documentclass{ws-jai}
\usepackage[flushleft]{threeparttable}
\usepackage{pdflscape}
\begin{document}

\catchline{}{}{}{}{} % Publisher's Area please ignore

\markboth{Jack Hickish, CASPER gang, etc.}{The Collaboration for Astronomy Signal Processing and Electronics Research in 2016.}

\title{The Collaboration for Astronomy Signal Processing and Electronics Research in 2016}

%\author{First Author$^\dagger$, Second Author$^\ddagger$, Third Author$^\ddagger$ and Fourth Author$^\S$}
\author{Jack Hickish$^\dagger$, Others$^\ddagger$}

\address{
$^\dagger$Radio Astronomy Laboratory, UC Berkeley, Berkeley, CA 94720, USA, jackh@astro.berkeley.edu\\
$^\ddagger$Group, Company, Address, City, State ZIP/Zone, Country\\
$^\S$Group, Company, Address, City, State ZIP/Zone, Country, fauthor@company.com
}

\maketitle

\corres{$^\dagger$Jack Hickish}

\begin{history}
\received{(to be inserted by publisher)};
\revised{(to be inserted by publisher)};
\accepted{(to be inserted by publisher)};
\end{history}

\begin{abstract}

The Collaboration for Astronomy Signal Processing and Electronics Research
(CASPER) has been working for a decade to reduce the time and cost of designing,
building and deploying new digital radio astronomy instruments.  Today,
CASPER-designed hardware powers some ?? scientific instruments worldwide, and
are used by scientists and engineers at ?? academic institutions.  In this paper
we summarize the current offerings of the CASPER collaboration, focussing on
currently-available and next-generation hardware.  We describe the ongoing state
of software development, as CASPER looks to support an ever-increasing selection
of off-the-shelf digital signal processing platforms.

\end{abstract}

\keywords{CASPER, digital signal processing, radio astronomy, instrumentation}


\section{Introduction}

Since the first digital instrument used in radio astronomy \citep{Weinreb} we
have seen a growing adoption of digital processing hardware as the foundation on
which radio telescopes are built.  Today, CPUs, GPUs, FPGAs and ASICs power
almost all of the world's radio telescopes, and our ability to do science has
become inextricably linked with our ability to perform digital computation.
With the capability of digital processing hardware scaling exponentially with
Moore's law, the ability to leverage current technology by reducing the
design-time of new instruments is critical in effective deployments of new radio
astronomy instruments.

The Collaboration for Astronomy Signal Processing and Electronics Research
(CASPER) puts \emph{time-to-science}, the time between conception of an
instrument and its deployment, as a central figure of merit in instrument
design. CASPER works to minimize time-to-science by working to develop and
support open-source, general-purpose hardware, software libraries and
programming tools which allow rapid instrument design, and straightforward
upgrade cycles.

With CASPER hardware and software now powering some ?? radio astronomy
instruments worldwide (see Table~\ref{table:casper-instruments}) including some
of the largest, most advanced telescopes ever built, such as the upcoming
MeerKat Array \citep{MeerKAT}, it is appropriate to document the state of the
Collaboration.

In this paper, we first summarize the design philosophy of CASPER in
Section~\ref{sec:CASPER-philosophy}.  In Section~\ref{sec:Hardware} we describe
currently available CASPER hardware offerings, including the range of digitizers
developed and supported by CASPER. Key to CASPER's success are the firmware
libraries and programming infrastructure provided by the collaboration, which we
overview in Section~\ref{sec:Software}.  In Section~\ref{sec:Deployments} we
document the extensive and wide-ranging applications to which CASPER hardware
and design-tools have been applied. Finally, we describe the future direction of
and challenges faced by the CASPER collaboration in Section~\ref{sec:Future},
with concluding remarks in Section~\ref{sec:Conclusions}.

\section{The CASPER Philosophy} \label{sec:CASPER-philosophy}

%% Jason Manley's PHD has a good section on Philosophy and Ethernet.

\subsection{Computing by the yard}

\subsection{Ethernet}

\subsubsection{Multicast \& Commensal Instruments}


\section{CASPER Hardware} \label{sec:Hardware}

\subsection{FPGA platforms}

\subsubsection{ROACH1}

%% Wesley New

\subsubsection{ROACH2}

%% Wesley New

\subsubsection{SKARAB}

%% Adam Isaacson

\subsubsection{SNAP}

\subsection{ADCs \& DACs}


\section{CASPER Software \& Programming Tools} \label{sec:Software}

\subsection{The CASPER Toolflow}

%% Wesley New

\subsection{JASPER Toolflow}

%% Adam Isaacson / Jack Hickish

\subsection{CASPER DSP Libraries}

%% Andrew Martens + anyone else keen to contribute

\emph{Just Another Signal Processing EnviRonment}


\section{CASPER Deployments} \label{sec:Deployments}

\newcommand{\rr}{\raggedright}
\newcommand{\tn}{\tabularnewline}
\newcommand{\ac}{\centering}
\begin{landscape}
\begin{table}
  \centering
  %\begin{tabular}{p{3cm} c p{4cm} p{4cm} p{4cm} p{2cm}}
  \begin{tabular}{p{3cm} c p{4cm} p{8cm} p{2cm}}
  %\ac Instrument & \ac Year & \ac Hardware & \ac Description & \ac CASPER Functionality & \ac References \tn
  \ac Instrument & \ac Year & \ac Hardware & \ac Description & \ac References \tn
  \hline
  %Academic Radio Interferometer (ARI) & 2012 & \rr 1xROACH1 1xADC2x1000-8 & \ac 21-cm dual-antenna interferometer for teaching purposes & \rr digitization, delay compensation, correlation & see email from Pedro Salas \\
  %pocketcorr & 2014 & \rr ROACH1 / ROACH2 / SNAP & \ac Multi-platform single-board correlator. & \rr digitization, channelisation, correlation & \footnote{\url{https://github.com/domagalski/pocketcorr}} \\
  %Leuschner Spectrometer & 2015 & \rr ROACH1, iADC  & \ac dual-polarization, 12 MHz, 8192 channel spectrometer. & \rr digitization, channelisation, power detection & \footnote{\url{http://w.astro.berkeley.edu/~domagalski/leuschner-radio/}} \\
  %BLAST-TNG & 2017 & \rr 5xROACH2, 5xMUSIC-DAC/ADC  & \ac 2.5~m Balloon-Borne Submillimeter Polarimeter & \rr MKID readout & \cite{galitzki2014balloon} \\
  %DSN Transient Observatory & 2016 & \rr 2xROACH1, 2xKATADC  & \ac Versatile signal processor for commensal astronomy during DSN data downlinks & \rr Kurtosis Spectrometer, pulse detection & Kuiper et al (in prep) \\
  %MeerKAT & Under Construction & \rr ROACH2 (SKARAB upgrade forthcoming)  & \ac "Facility Instrument" capable of producing various data products over 856~MHz bandwidth & \rr 32k channel, 64 dual-pol antenna correlator. Beamformer. Transient buffer. & MeerKAT CBF Requirement Spec (private) \\
  %KAT7 & 2010 & \rr 16xROACH1  & \ac 7 dual-pol antenna full-stokes correlator & \rr channelization. correlation & \cite{Foley01082016}, Manley thesis \\
  %RATTY & 2012 & \rr 1xROACH1  & \ac Transient / RFI Monitor for SKA-SA site monitoring& \rr  & \cite{Foley01082016}, Manley thesis \\
  Academic Radio Interferometer (ARI) & 2012 & \rr 1xROACH1 1xADC2x1000-8 & \ac 21-cm dual-antenna interferometer for teaching purposes & see email from Pedro Salas \\
  pocketcorr & 2014 & \rr ROACH1 / ROACH2 / SNAP & \ac Multi-platform single-board FX correlator. Used in HYPERION deployment and PAPER testing & \footnote{\url{https://github.com/domagalski/pocketcorr}} \\
  Leuschner Spectrometer & 2015 & \rr ROACH1, iADC  & \ac dual-polarization, 12 MHz, 8192 channel spectrometer for UC Berkeley's Leuschner Radio Observatory & \footnote{\url{http://w.astro.berkeley.edu/~domagalski/leuschner-radio/}} \\
  BLAST-TNG & 2017 & \rr 5xROACH2, 5xMUSIC-DAC/ADC  & \ac 2.5~m Balloon-Borne Submillimeter Polarimeter with CASPER MKID readout system & \cite{galitzki2014balloon} \\
  DSN Transient Observatory & 2016 & \rr 2xROACH1, 2xKATADC  & \ac Versatile signal processor for commensal astronomy during DSN data downlinks, featuring Kurtosis Spectrometer and pulse detection & Kuiper et al (in prep) \\
  MeerKAT & Under Construction & \rr ROACH2 (SKARAB upgrade forthcoming)  & \ac "Facility Instrument" capable of producing various data products over 856~MHz bandwidth. Modes include 32k channel, 64 dual-pol antenna correlator, beamformer, transient buffer. & MeerKAT CBF Requirement Spec (private) \\
  KAT7 & 2010 & \rr 16xROACH1  & \ac 7 dual-pol antenna full-stokes FX correlator & \cite{Foley01082016}, Manley thesis \\
  RATTY & 2012 & \rr 1xROACH1  & \ac Transient / RFI Monitor for SKA-SA site monitoring. & \cite{Foley01082016}, Manley thesis \\
  HOLMES & Deployment 2018 & \rr 35xROACH2, MUSIC-ADC/DAC & \ac Electron Neutrino Mass measurement with CASPER-based microwave SQUID readout system. &  \cite{Alpert2015, Ferri2016179} \\
  GAVRT & 2009-2012 & \rr 8xiBOB, 16xiADC, BEE2 & \ac 8~GHz instantaneous bandwidth transient capture buffer with real-time incoherend dedispersion trigger.  & \cite{jon10, doi:10.1117/12.856642} \\
  cycSpec & 2012 & \rr ROACH1/ADC83000 (Arecibo), ROACH2/ADC5G (GBT) & \ac Real-time cyclic spectrometer, deployed at Arecibo and GBT on consecutive generations of hardware. 128~MHz bandwidth CASPER-based overlapping filterbank used to feed GPU processors.  & Jones et al (in prep) \\
  Columbia MKID readout & 2012 & \rr initially ROACH1, upgraded to ROACH2 & \ac MKID readout system with CASPER-based tone generation, digitization and coarse channelization. Feeds non-CASPER HPC processors. & \cite{mccarrick_2014} \\
  GRASP & 2016 & \rr 1xROACH1, 1xQUAD ADC & \ac 100 MHz bandwidth full-stokes spectrometer for the Gauribidnaur Radio Solar spectro-Polarimter. & Indrajit \\
  AMiBA Correlator Upgrade & 2016 & \rr 8xROACH2, 14xADC5G & \ac 7 dual-pol antenna 4.48~GHz bandwidth FX correlator & Homin \\
  ALMA Phased Array & 2014 & \rr 8xROACH2 & \ac Time-tagging, ethernet packetization and VDIF (VLBI) formatting & \cite{2012evn..confE..53A} \\
  AMI Correlator Upgrade & 2015 & 18xROACH2, 36xADC5G & 5~GHz, 4096 channel FX Correlator for the Arcminute Microkelvin Imager arrays. & \cite{Zwart21122008}, Hickish et al. (in prep) \\ 
  SWARM Correlator & & & & \\
  Medicina FFTT & 2014 & 3xROACH1, 1xADC64 & A multi-board digitization, channelisation, beamforming correlation system, used to demonstrate direct-imaging on the BEST-2 Array & \cite{Foster11042014} \\
  HIPSR & 2014 & 13xROACH1, 13xiADC & 400~MHz bandwidth 8192 channel spectrometer and high time-resolution system for the Parkes multibeam receiver & \footnote{\url{http://telegraphic.github.io/hipsr/overview.html}} \\
  MITEoR & 2014 & 4xROACH2 4xADC64 & 50~MHz bandwidth, 128 dual-polarization antenna FX correlator, used to investigate spatial-FFT correlation methods & \cite{2014MNRAS.445.1084Z} \\
  \end{tabular}
\end{table}
\end{landscape}

\subsection{MeerKAT}

%% Ask Ruby van Rooyen

\section{Future Directions \& Challenges} \label{sec:Future}

\subsection{Hardware Design Challenges}

\subsubsection{Timing Closure}

\subsubsection{High speed memories}

%% HMC 

\subsection{Support of off-the-shelf hardware}

\subsection{CPU/GPU programming/data-transport}

\subsection{Design re-use}

\subsection{Observatory Integration}


\section{Conclusions} \label{sec:Conclusions}


\bibliographystyle{ws-jai} \bibliography{casper-2016}

\end{document} 
